\section[Einleitung (Schenkel)]{Einleitung}
\label{einleitung}
\begin{quote}
"[...] Seit der Einführung der Telefonkarte im Jahr 1985 in Deutschland sind Chipkarten ein
Massenartikel. [...] "\cite{spitz11}
\end{quote}

Laut Spitz begann die weite Verbreitung von Chipkarten etwa ein Jahrzehnt nachdem erste Patente
zu 'Identitätskarten mit Mikroprozessor' eingereicht wurden. Seit dieser Entwicklung
werden Chipkarten in unterschiedlichen Formfaktoren, Funktionsumfang und Einsatzgebieten
verwendet.

Chipkarten, die oft auch als SmartCard oder Integrated Circuit Card bezeichnet werden existieren
allgemein zu dem Zweck eine Identität beziehungsweise Berechtigung (z.B. auch in Form von Guthaben) nachzuweisen.
Die beiden populärsten Vertreter sind die elektronische Krankenversichertenkarte im Gesundheitsbereich
sowie die \ac{SIM}-Karte im Bereich des Mobilfunk.

Eine bestimme Ausprägung einer Chipkarte wird, je nach Zweck, einem Mobilfunk\-vertrags- beziehungsweise Versicherungsteilnehmer
ausgeliefert, damit dieser sich über selbige 'ausweisen' kann. Der reine Identitätsnachweis ist
allerdings nicht die einzige Funktion, die eine Chipkarte liefern kann. Neben reinen Speicherkarten
existieren auch Prozessorkarten, die über das reine liefern (meist sensitiver) Informationen hinaus,
Algorithmen integrieren. Diese Algorithmen dienen unter anderem zur Sicherstellung der Geheimhaltung
sensitiver Informationen.

Der Ursprung der Chipkarten im Bereich des Mobilfunk liegt bei der 1985 in Deutschland eingeführten
Telefonkarte. Sie setzten sich aufgrund ihrer Robustheit und Fälschungs\-sicherheit gegen ebenfalls
geprüfte Magnetkarten sowie Lochkarten durch. Als sich Ende der 90er-Jahre der Mobilfunk stark
ausbreitete, sank die Nachfrage nach Telefonkarten und wurde durch \ac{SIM}-Karten abgelöst.
Sowohl bei Prepaid- als auch bei gewöhnlichen Telefonverträgen ist es notwendig, dass der Teilnehmer
eine Chipkarte vom Provider erhält und diese in sein Mobiltelefon einsetzt.

Diese \ac{SIM}-Karte ist optional mit einer Zahlenkombination (\ac{PIN}) vor Zugriff von Fremden geschützt.
Ist die Karte erst freigeschaltet, kann das Mobiltelefon den Authentifizierungsprozess (unter Verwendung der
auf der Karte integrierten Kryptoalgorithmen) mit dem Provider initiieren. Nach erfolgreicher Durchführung
dieses Prozesses, also Identifizierung durch den Provider, ist der Mobilfunknehmer dazu berechtigt
im Rahmen seines Vertrages das Mobilfunknetz zu benutzen.

Welche Applikationen in welcher Form auf der Prozessorchipkarte tatsächlich laufen, ist von Hersteller zu Hersteller unterschiedlich.
Eine Standardisierung findet durch dir Norm \textit{ISO 7816} statt. Sie definiert vor allem physikalische
und elektrische Eigenschaften von Chipkarten.

Mit dieser Grundlage entstand die Idee zu dieser Studienarbeit.

\subsection[Idee zur Arbeit (Schenkel)]{Idee zur Arbeit}
\label{idee-arbeit}
Ziel dieser Studienarbeit ist es die Freischaltung eines Netzzugangspunktes unter Verwendung einer 
gebräuchlichen \ac{SIM}-Karte zu realisieren. Der Vorteil hierbei ist, dass nahezu jeder
Vertragsnehmer bei einem Provider für Breitbandanschlüsse über einen Mobilfunkvertrag verfügt.
Sogar Senioren besitzen in der Regel ein Handy oder gar Smartphone und somit zwingend auch eine
\ac{SIM}-Karte. Nach einer Umfrage durch Pro Senectute\footnote{\url{https://www.prosenectute.ch/de.html}}
im Jahr 2014 sind dies über 70\% der ab 65-Jährigen.

Nicht nur als Ergänzung des heimischen Anschlusses, sondern auch als zuverlässige und sichere Lösung
für Hotels oder ähnliche Institutionen ist dieses Setup denkbar.

Es soll gezeigt werden, dass dies durch Kombination bereits bestehender technischer Mittel und
eigener Implementierung umgesetzt werden kann. Grundlage dieser Umsetzung is eine von Herrn Prof. Müller
bereitgestellte \ac{SIM}-Karte, deren Geheimnisse bekannt sind. Weiterführend wird ein Kleincomputer
(Raspberry Pi\footnote{\url{https://www.raspberrypi.org/}}) als Endgerät (\ac{UE}) sowie eine virtuelle
Maschine zur Simulation des Netzproviders (\ac{AuC}) eingesetzt. 

Der Raspberry Pi bietet sich besonders für dieses Vorhaben an, da dieser preisgünstig in der Anschaffung
und sehr gut in Bezug auf Erweiterbarkeit konstruiert ist. Das Lesen der \ac{SIM}-Karte
wird durch einen entsprechenden Kartenleser via USB-Technologie ermöglicht. Um möglichst nah
an der realen Umgebung zu bleiben soll die Verbindung zwischen \ac{UE} und \ac{AuC} über eine
\ac{PPPoE}-Verbindung (\ac{PPP} über Ethernet) realisiert werden.

\clearpage