\section{Diskussion}
\label{diskussion}

	\subsection{Alternative Vorgehensweisen}
	\label{diskussion-alternative}
	
	Die Arbeit ließ einiges an Spielraum besonders die Art der Umsetzung. Deshalb soll in
	diesem Kapitel aufgezeigt werden, was alternativ möglich gewesen wäre.
	
	Eine Möglichkeit wäre sicherlich gewesen für beide Seiten, also UE und AuC, die selbe
	Programmiersprache zu verwenden. So hätte beides mit C geschrieben werden können,
	jedoch wäre der Aufwand ein größerer geworden. Das liegt daran, dass für C keine
	Wrapper-Klasse in dem Umfang von pyscard (vgl. Kapitel \Verweis{pyscard}) existiert.
	Somit müssten einige Dinge, die durch pyscard und osmo-sim-auth abgewickelt wurden,
	selbst programmiert werden. Aber die Funktionalität wäre grundsätzlich durch PCSClite
	gegeben die USIM über C anzusprechen. \\
	Umgekehrt wäre die Möglichkeit gewesen den Netzprovider ebenfalls in Python zu
	implementieren. Die Fähigkeiten von Python wären für die Aufgabe ausreichend gewesen,
	zumal keine Optimierung auf Größe oder Geschwindigkeit des Programms gelegt wurde.
	Durch seine Hardwarenähe hat C in beiden Bereichen gegenüber Python einen Vorteil,
	aber diese war nicht gefragt. Jedoch hätte es bei Python das Problem gegeben, dass
	die vorhanden Kenntnisse sehr gering waren im Gegensatz zu C. Hier wäre also zu Beginn
	ein größerer Aufwand entstanden die Sprache zu lernen. \\
	Da das Programm zur Steuerung der SIM-Karte sehr geil ist, ist es jedoch kein Widerspruch
	hier die Zeit für die Grundkentnisse in Python zu investieren, da sie vermutlich geriger waren
	als der Zeitaufwand, der für eine Implementierung in C nötig gewesen wäre.
	
	Abschließend ist zu sagen, dass die Liste der Sprachen aus der gewählt werden konnte ziemlich
	umfangreich ist, aber mit pyscard existiert ein guter Wrapper für PCSClite und es ergab Sinn
	ein Projekt was mit Hardware kommuniziert auch mit einer hardwarenahen Sprache umzusetzen.	
	
	\subsection{Ausblick}
		\subsubsection{Verbesserungen}
		Wie in Kapitel \Verweis{ergebnis-tests} angesprochen wurde, sind alle gesetzten Ziele
		erreicht worden, aber es gab hier und da einige Probleme, die aufgetreten sind und
		manche von denen wären vermutlich vermeidbar gewesen. Dazu zählen vor allem Fehler
		bei der Entwicklung des AES- und Milenage-Algorithmuses.
		
		Lange bestand eine unterschiedliche Meinung darüber, wo der Unterschied von UMTS-
		und GSM-Authentifizierung lag. Und trotz vermehrter kurzer Unterhaltungen in denen dies
		offensichtlich wurde, hat es lange gedauert bis ein längeres Gespräch stattfand in dem
		gemeinsam geklärt wurde, wie sich beide unterscheiden. Wäre dieser Diskurs schon früher
		geführt worden, wäre eine erste Implementierung mit pysim vermutlich nicht geschehen,
		sondern die Suche wäre weiter gegangen und schon früher auf osmo-sim-auth gefallen.
		
		Der andere große Punkt war, dass nicht bekannt war, wie und wer den Wert von SQN
		verantwortet. Der Wert tauchte zwar immer wieder auf, aber in den Dokumenten, die gelesen
		wurden war dieser Wert im Gegensatz zu allen anderen nicht näher spezifiziert. Deshalb
		wurde zu Beginn angenommen, dass der Netzprovider SQN entscheidet und durch den AUTN
		die SIM-Karte diesen vermittelt bekommt. Es stellte sich jedoch raus, dass SQN von der USIM
		verwaltet wird und es sehrwohl eine Methode zur Berechnug der SQN gibt, aber es wurde
		an den falschen Stellen danach gesucht und versuchte deshalb kurz vor Projektabschluss
		nochmal Probleme. Ein Lesen der Referenzen der Milenage-Spezifikation 35.205 vom 3GPP
		hätte direkt zu dem Dokument 33.102 geführt, in der auch die SQN erläutert wurde. Hier gilt es beim
		nächsten Mal nicht erst Google für die Suche zu bemühen sondern das zu durchsuchen, was
		schon existiert.
		
		\subsubsection{Mögliche Erweiterungen}
		
		
		\subsubsection{Andere Einsatzgebiete ???}
	\subsection{Sicherheit / Angriffe}
		Es wurde keine Rücksicht genommen auf ..., weil...
\clearpage