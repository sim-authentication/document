\clearpage

\section{Diskussion}
\label{diskussion}

	\subsection{Alternative Vorgehensweisen}
	\label{diskussion-alternative}
	
	Die Arbeit ließ einiges an Spielraum besonders die Art der Umsetzung. Deshalb soll in
	diesem Kapitel aufgezeigt werden, was alternativ möglich gewesen wäre.
	
	Eine Möglichkeit wäre sicherlich gewesen für beide Seiten, also UE und AuC, die selbe
	Programmiersprache zu verwenden. So hätte beides mit C geschrieben werden können,
	jedoch wäre der Aufwand ein größerer geworden. Das liegt daran, dass für C keine
	Wrapper-Klasse in dem Umfang von pyscard (vgl. Kapitel \Verweis{pyscard}) existiert.
	Somit müssten einige Dinge, die durch pyscard und osmo-sim-auth abgewickelt wurden,
	selbst programmiert werden. Aber die Funktionalität wäre grundsätzlich durch PCSClite
	gegeben die USIM über C anzusprechen. \\
	Umgekehrt wäre die Möglichkeit gewesen den Netzprovider ebenfalls in Python zu
	implementieren. Die Fähigkeiten von Python wären für die Aufgabe ausreichend gewesen,
	zumal keine Optimierung auf Größe oder Geschwindigkeit des Programms gelegt wurde.
	Durch seine Hardwarenähe hat C in beiden Bereichen gegenüber Python einen Vorteil,
	aber diese war nicht gefragt. Jedoch hätte es bei Python das Problem gegeben, dass
	die vorhanden Kenntnisse sehr gering waren im Gegensatz zu C. Hier wäre also zu Beginn
	ein größerer Aufwand entstanden die Sprache zu lernen. \\
	Da das Programm zur Steuerung der SIM-Karte sehr geil ist, ist es jedoch kein Widerspruch
	hier die Zeit für die Grundkenntnisse in Python zu investieren, da sie vermutlich geringer
	waren als der Zeitaufwand, der für eine Implementierung in C nötig gewesen wäre.
	
	Abschließend ist zu sagen, dass die Liste der Sprachen aus der gewählt werden konnte
	ziemlich umfangreich ist, aber mit pyscard existiert ein guter Wrapper für PCSClite und
	es ergab Sinn ein Projekt was mit Hardware kommuniziert auch mit einer hardwarenahen
	Sprache umzusetzen.	
	
	\subsection{Ausblick}
		\subsubsection{Verbesserungen}
		Wie in Kapitel \Verweis{ergebnis-tests} angesprochen wurde, sind alle gesetzten
		Ziele erreicht worden, aber es gab hier und da einige Probleme, die aufgetreten
		sind und manche von denen wären vermutlich vermeidbar gewesen. Dazu zählen vor
		allem Fehler bei der Entwicklung des AES- und Milenage-Algorithmuses.
		
		Lange bestand eine unterschiedliche Meinung darüber, wo der Unterschied von UMTS-
		und GSM-Authentifizierung lag. Und trotz vermehrter kurzer Unterhaltungen in
		denen dies offensichtlich wurde, hat es lange gedauert bis ein längeres Gespräch
		stattfand in dem gemeinsam geklärt wurde, wie sich beide unterscheiden. Wäre
		dieser Diskurs schon früher geführt worden, wäre eine erste Implementierung mit
		pysim vermutlich nicht geschehen, sondern die Suche wäre weiter gegangen und
		schon früher auf osmo-sim-auth gefallen.
		
		Der andere große Punkt war, dass nicht bekannt war, wie und wer den Wert von SQN
		verantwortet. Der Wert tauchte zwar immer wieder auf, aber in den Dokumenten,
		die gelesen wurden war dieser Wert im Gegensatz zu allen anderen nicht näher
		spezifiziert. Deshalb wurde zu Beginn angenommen, dass der Netzprovider SQN
		entscheidet und durch den AUTN die SIM-Karte diesen vermittelt bekommt. Es
		stellte sich jedoch raus, dass SQN von der USIM verwaltet wird und es sehr wohl
		eine Methode zur Berechnung der SQN gibt, aber es wurde an den falschen Stellen
		danach gesucht und versuchte deshalb kurz vor Projektabschluss nochmal Probleme.
		Ein Lesen der Referenzen der Milenage-Spezifikation 35.205 vom 3GPP hätte direkt
		zu dem Dokument 33.102 geführt, in der auch die SQN erläutert wurde. Hier gilt
		es beim nächsten Mal nicht erst Google für die Suche zu bemühen sondern das zu
		durchsuchen, was schon existiert.
		
		\subsubsection{Mögliche zukünftige Projekte}
		Die Authentifizierung und der Schlüsselaustausch von UMTS ist noch umfangreicher
		als was in diesem Projekt umgesetzt wurde. Da es in diesem Projekt allein um die
		Authentifizierung ging, wurde die Berechnung von CK und IK vernachlässigt. Dies
		wäre in einem weiteren Schritt eine Möglichkeit. Dadurch könnten dann auch
		Nachrichten zwischen Nutzergerät und Authentifizierungsstelle verschickt werden.
		Noch ein Schritt weiter wäre eventuell eine weitere SIM-Karte, die sich beim
		Netzprovider authentifiziert und dann mit der jetzigen USIM kommuniziert.

		Für diesen Schritt müsste jedoch vorher das Programm auf dem AuC angepasst
		werden und eine Datenbank erhalten, in der die unterschiedlichen Parameter für
		die einzelnen USIM's abgespeichert werden.

		Generell erlaubt der aktuelle Prozess nur USIM-Karten. Hier wäre zu überlegen
		für SIM-Karten auf dem GSM-Standard ebenfalls die
		Authentifizierung zu ermöglichen. Der Milenage-Algorithmus ermöglicht die
		Abwärtskompatibilität, weshalb kein neuer Standard verstanden werden muss und
		der zusätzliche Programmieraufwand nicht sehr groß sein wird.
	
		Generell sollte auch der Programmcode optimiert werden. An einigen Stellen
		können Programmteile in eigne Funktionen ausgegliedert werden, um Redundanzen
		zu vermeiden und das Programm selbst im Speicherplatzverbrauch zu reduzieren. Des
		weiteren wird schon jetzt eine ähnliche Aufgabe durch rotWord und shiftRows
		gegeben, weshalb shiftRows einfach rotWord implementieren könnte. \\
		Neben der Reduzierung von Redundanzen kann überlegt werden, wo sich die
		Geschwindigkeit des Programms optimieren lässt. Am Ende ist den Autoren zum
		Beispiel aufgefallen, dass es für die Umwandlung des Schlüssels und des
		Textblocks in das Array eine schnellere Variante gibt als die aktuelle. Dabei
		ist der Code dann doch zeilenweise in dem Array gespeichert und nicht wie aktuell
		spaltenweise. Damit müssten auch die einzelnen Transformations-Funktionen
		umgebaut werden, was in Anbetracht der Zeit in diesem Projekt nicht mehr umsetzbar
		war. \\
		So gibt es aber vermutlich noch einige Stellen an denen eine Optimierung unter dem
		Aspekt Performanz und wahrscheinlich auch Speicherbedarf möglich ist.
	
		Als nächster Punkt für zukünftige Erweiterungen bestünde die Möglichkeit den
		Start der Authentifizierung automatisiert zu erkennen. So bietet PCSClite die
		Möglichkeit eine Meldung zu erhalten, wenn in das Kartenlesegerät die SIM-Karte
		gesteckt wird, beziehungsweise auch wenn diese wieder entfernt wird. Damit
		könnte der Server dauerhaft laufen und der Raspberry Pi würde sich ohne den
		manuellen Start durch einen Nutzer beim Netzprovider registrieren.

		Als letzten sinnvollen Punkt wird das Abbilden der realen Struktur gesehen. Das
		bedeutet, dass eine weitere virtuelle Maschine als SGSN benötigt würde. Außerdem
		sollten dann auch die jeweiligen MAC-Werte und der RES an der richtigen Stelle
		überprüft werden mit der richtigen Reaktion durch die einzelnen Kommunikationspartner.

	\subsection{Sicherheit / Angriffe}
		Es wurde keine Rücksicht genommen auf ..., weil...
