\section{Tätigkeit}
	\subsection{Erstellen des Netzproviders}
		\subsubsection{Ubuntu}
		Wie in \Verweis{subsec:ubuntu} beschrieben wird Version 14.04 LTS
		der Linuxdistribution Ubuntu installiert. Das bei der Herstellerwebsite\footnote{\url{http://releases.ubuntu.com/14.04/}},
		für Serverbetrieb erhältliche Image wird in einer virtuellen Maschine installiert.
		Diese weist folgende Eigenschaften auf:

		\begin{tabularx}{\textwidth}{|l|X|}
    	\hline
      		\textbf{Gerät} & \textbf{Typ} \\
    	\hline
    	\hline
    		CPU & 1x 64 Bit \\
    	\hline
    	\hline
    		RAM & 1x 1024 MB \\
    	\hline
    	\hline
    		HDD & 1x 10 GB \\
    	\hline
    	\hline
    		NIC & 2x Bridged (1x auf enp0s25; 1x auf wlp3s0) \\
    	\hline
    	\end{tabularx}

    	Aufgrund der Tatsache dass nur ein einziger Client übder den Zugang von \textit{PPPoE}
    	zu erwarten ist, sind die Anforderungen an CPU, Arbeits- und Festplattenspeicher
    	recht gering. Wichtig ist das Vorhandensein zweier Netzwerkschnittstellen, die
    	auf jeweils eine physikalisch Schnittstelle geleitet werden. Dies ist notwendig,
    	da sowohl die \ac{MS} über eine Ethernetverbindung mit der virtuellen Maschine angebunden sein,
    	als auch eine Dauerhafte Internetverbindung mit der verbleibenden Schnittstelle
    	aufrecht erhalten werden muss. So ist es möglich nach erfolgreicher Authentisierung
    	des \ac{MS} den Zugang zum Internet freizuschalten.

    	\subsubsection{PPPoE}
    	kann nach der vollständigen Ubuntuinstallation durchgeführt werden. Hierzu wird die aktuelle
    	Version 3.11 auf der Herstellerwebsite\footnote{\url{https://www.roaringpenguin.com/products/pppoe}}
    	heruntergeladen und auf dem Server entpackt. Zusätzlich wird das Ubuntu-Paket zur Unterstützung
    	des Point-To-Point-Protocol (als \textit{ppp} erhältlich) benötigt. Sind diese beiden Grundvoraussetzungen
    	abgedeckt wird der Roaring Penguin PPPoE-Server ohne spezielle Änderungen kompiliert.

    	In der Datei \textit{/etc/ppp/pppoe-server-options} werden die Parameter gesetzt. Anzugeben
    	sind die DNS-Server sowie die CHAP-Authentifizierung. Der User für die Authentifizierung
    	wird unter \textit{/etc/ppp/chap-secrets} definiert.

    	Der zu verwendende Adresspool an IP-Adressen ist unter \textit{/etc/ppp/allip} anzugeben.

    	Abschließend fehlt noch die Deklaration der zu verwendenden Netzwerkschnittstelle (unter \textit{/etc/network/interfaces}),
    	die zum Betrieb mit PPPoE eingesetzt werden soll. Ihr muss eine valide IP-Adresse aus dem definierten
    	Pool sowie eine Subnetzmaske zugewiesen werden.

    	Alle Konfigurationsdateien befinden sich im Anhang. %TODO: #edit Anhang pflegen

    	Damit die Internetverbindung nach erfolgreicher Authentifizierung des \ac{MS} auch
    	korrekt von Provider freigeschaltet und weitergeleitet wird, muss eine IPTables-Regel
    	eingerichtet werden:

    	\begin{lstlisting}[language=bash]
  			$ iptables -t nat -A POSTROUTING -s 192.168.178.0/24 -o enp0s25 -j MASQUERADE
		\end{lstlisting}

		Sie nimmt den Verkehr auf dem Device \textit{enp0s25} an und maskiert (\ac{NAT}) diesen
		für den weiteren Betrieb (wie es bei einem realen Provider auch wäre). Dies geschieht
		für eingehenden sowie ausgehenden Verkehr. Die auf den zweiten Netzwerkadapter geleitete
		Netzwerkschnittstelle bleibt mit Defaultwerten konfiguriert und fungiert als Zugangspunkt
		zum Internet.

		Nun kann der Server auf der gewünschten Netzwerkschnittstelle mit einer eigenen IP-Adresse
		gestartet und genutzt werden:
		\begin{lstlisting}[language=bash]
		pppoe-server -C isp -L 192.168.178.254 -p /etc/ppp/allip -I enp0s25
		\end{lstlisting}

		\subsubsection{Implementierung Milenage}
		\subsubsection{Implementierung AES}
			\paragraph{'Unter'-Algorithmen}
	\subsection{Erstellen des UE}
		\subsubsection{Aufsetzen des Pis}
			\paragraph{rasbian}
			\paragraph{pysim}
	\subsection{Integration mit PPPoE}
		 \subsubsection{roaring penguin pppoe}

\clearpage