\section{Ergebnis}
	\subsection{Tests}
		\subsubsection{AES}
		\subsubsection{Milenage}
		\subsubsection{SIM-Karte+pySIM}
		\subsubsection{HeuKel}
	\subsection{Fehler}
		\subsubsection{Einrichtung verwendeter Werkzeuge}
			\paragraph{Raspbian} konnte ohne Probleme installiert und in
			Betrieb genommen werden (vgl. \Verweis{subsubsec:installpi}). Der einzige Zwischenfall ist auf eine
			defekte SD-Speicherkarte zurückzuführen. Nach wenigen Wochen
			des Betriebes ist diese ausgefallen, was ein starten des
			Raspberry Pis unmöglich machte. Ebenfalls konnten die auf der
			Speicherkarte enthaltenen Daten nicht wieder hergestellt werden.

			Aus diesem Grund war es notwendig das Betriebssystem erneut aufzusetzen.
			Eingeleitete Gegenmaßnahme für zukünftige Ausfälle war ein regelmäßiges Backup
			nach Änderungen am System. Via \textit{dd} wurde immer blockweise ein komplettes Abbild
			des aktuellen Systems auf externen Festplattenspeicher übertragen.

			Es blieb bei einem einzigen Hardwareausfall der SD-Karte.

			\paragraph{Ubuntu} konnte ohne Probleme installiert sowie betrieben werden.
			Schon in der Standardinstallation des Server-Images können Netzwerkschnittstellen,
			von Virtualbox bereitgestellt und direkt konfiguriert bzw. auch genutzt werden.

			\paragraph{Milenage-Prüfsoftware} wurde während der Implementierungsphase des
			Milenage-Algorithmus mit Empfehlung von Herrn Prof. Müller eingesetzt.
			Durch das Eintragen der gewählten sowie festen (geheimen) Werte konnte
			die eigene Implementierung des Milenage-Algorithmus überprüft werden.

			Besonders hilfreich ist die Ausgabe aller Zwischenergebnisse von Teilfunktionen
			\textit{f1}-\textit{f5} sowie \textit{AUTN}, \textit{Kc} und \textit{SRES} auf
			den Seiten beider Teilnehmer.

			\paragraph{pysim} stellte sich während der Implementierungsphase als
			problematisch heraus, da es nicht über alle der benötigten Funktionalitäten
			verfügt. Den Verfassern dieser Studienarbeit war bei der ersten Auswahl
			der Pythonbibliothek nicht klar, ob die reine Unterstützung für \ac{GSM}-Karten
			ausreicht.

			Nachdem sich herausstellte, dass \ac{UMTS}-Kartensupport ebenfalls notwendig ist, fiel die
			Wahl auch \textit{osmo-sim-auth} von Osmocom (vgl. \Verweis{subsec:osmosim}). Über diese Bibliothek konnten
			alle Funktionen auf der Seite des \ac{MS} in der Sprache Python implementiert werden.
		\subsubsection{Fehleinschätzungen}
\clearpage