\section{Ergebnis}
	\subsection{Tests}
		\subsubsection{AES}
		\subsubsection{Milenage}
		\subsubsection{SIM-Karte+pySIM}
		\subsubsection{HeuKel}
	\subsection{Fehler}
		\subsubsection{Einrichtung verwendeter Werkzeuge}
			\paragraph{Raspbian} konnte ohne Probleme installiert und in
			Betrieb genommen werden (vgl. \Verweis{subsubsec:installpi}). Der einzige Zwischenfall ist auf eine
			defekte SD-Speicherkarte zurückzuführen. Nach wenigen Wochen
			des Betriebes ist diese ausgefallen, was ein starten des
			Raspberry Pis unmöglich machte. Ebenfalls konnten die auf der
			Speicherkarte enthaltenen Daten nicht wieder hergestellt werden.

			Aus diesem Grund war es notwendig das Betriebssystem erneut aufzusetzen.
			Eingeleitete Gegenmaßnahme für zukünftige Ausfälle war ein regelmäßiges Backup
			nach Änderungen am System. Via \textit{dd} wurde immer blockweise ein komplettes Abbild
			des aktuellen Systems auf externen Festplattenspeicher übertragen.

			Es blieb bei einem einzigen Hardwareausfall der SD-Karte.

			\paragraph{Ubuntu} konnte ohne Probleme installiert sowie betrieben werden.
			Schon in der Standardinstallation des Server-Images können Netzwerkschnittstellen,
			von Virtualbox bereitgestellt und direkt konfiguriert bzw. auch genutzt werden.

			\paragraph{Milenage-Prüfsoftware} wurde während der Implementierungsphase des
			Milenage-Algorithmus mit Empfehlung von Herrn Prof. Müller eingesetzt.
			Durch das Eintragen der gewählten sowie festen (geheimen) Werte konnte
			die eigene Implementierung des Milenage-Algorithmus überprüft werden.

			Besonders hilfreich ist die Ausgabe aller Zwischenergebnisse von Teilfunktionen
			\textit{f1}-\textit{f5} sowie \textit{AUTN}, \textit{Kc} und \textit{SRES} auf
			den Seiten beider Teilnehmer.

		\subsubsection{Fehleinschätzungen}
		Die meisten und gravierendsten Fehleinschätzunden bzw. Probleme
		wurden während der Inbetriebnahme der Karte sowie der Entwicklung
		der Softwarekomponenten aufgedeckt.

		\paragraph{pysim}
		Die erste Fehleinschätzung war der Einsatz des Tools pysim, welches
		primär zu ersten Test in der Kommunikation mit der vorliegenden
		SIM-Karte angedacht war. Es stellte sich während der Implementierungsphase
		als problematisch heraus, da es nicht über alle der benötigten 
		Funktionalitäten verfügt. Zwar konnte mit pysim ein erster
		Kommunikationsversuch erfolgreich gestartet und abgeschlossen werden,
		nicht jedoch eine Authentifizierung initiiert werden. Dies lag
		vor allem, dass die beiden Verfassern dieser Studienarbeit sich nicht
		sicher waren welche Art der Authentifizierung angewandt werden muss.
		Die für eine GSM- oder UMTS-fähige SIM-Karte.

		Bei ersten Tests mittels pysim wurde schnell deutlich, dass es
		sich um eine UMTS-Authentifizierung handeln muss. Die vorliegende
		Python-Bibliothek pysim ist jedoch nicht dazu in der Lage diese
		Art von Authentifizierung zu initiieren. Wie unter \Verweis{subsec:pysim}
		genauer erläutert konzentriert sich pysim von Kevin Prince auch
		eher auf die Programmierung entsprechender SIM-Karten, statt
		dem universellen auslesen dieser.

		Nachdem sich herausstellte, dass \ac{UMTS}-Kartensupport ebenfalls 
		notwendig ist, fiel die Alternativwahl
		auf die Bibliothek \textit{osmo-sim-auth} eingesetzt
		(vgl. \Verweis{subsec:osmosim}). Über diese Bibliothek konnten
		alle Funktionen auf der Seite des \ac{MS} in der Sprache Python
		implementiert werden.

		\paragraph{AUTN} Unsicherheiten enstanden auch bei ersten Testdurchläufen
		mit \textit{osmo-sim-auth}. Da zu diesem Zeitpunkt noch keine 
		vollständige Implementierung zur Generierung aller benötigten
		Werte existierte, wurde die Zusammensetzug für den
		Parameter \textit{AUTN} zufällig gesetzt. Dies resultierte jedoch
		darin, dass die SIM-Karte einen Fehler zurück lieferte. Mit
		Analyse des Fehlercodes \textit{SW 98 62} wurde schnell klar,
		dass die Zusammensetzung des Parameters \textit{AUTN} zwingend
		den Vorgaben entsprechen muss, um korrekt zu funktionieren.

		Nach der Fertigstellung der Implementierung konnte der Test erneut
		durchgeführt werden. Dieses mal mit Erfolg. Die SIM-Karte lieferte
		einen generierten Wert zurück: \textit{AUTS}.

		\paragraph{Resynchronisation der SQN} Während die verbleibende
		Implementierungsphase weitestgehend wie geplant verlief, zeichnete
		sich kurz vor Abschluss der Phase ein weiteres Defizit im Code ab.

		Mit der bereits erläuterten Milenage-Prüfsoftware konnten fast
		alle berechneten, übertragenen sowie von der SIM-Karte empfangenen
		Werte korrekt verifiziert werden. Ausgehend von den festegelegten
		Geheimnissen \textit{K}, \textit{OP(c)}, dem ausgetauschten \textit{RAND}
		über die berechneten Variablen \textit{f1-f5} bis hin zu \textit{Kc} sowie
		\textit{SRES} auf beiden teilnehmenden Seiten.
		Der Vergleich dieser Werte deutete darauf hin, dass die Implementierung
		korrekt sein muss.

		Problematisch war jedoch die Integration des Codes in die bereitstehende
		Umgebung. Während die Übertragung der Werte plangemäß verlief, antwortete
		die SIM-Karte nicht mit den erwarteten Vektoren \textit{Ck}, \textit{Ik}
		und \textit{Kc}, sondern lediglich mit einem einzigen Wert. Dieser
		stellte sich nach wiederholter Recherche in den 3GPP-Dokumenten als
		\textit{AUTS} heraus, welcher zur Resynchronisation der \textit{SQN}
		benötigt wird (vgl. \Verweis{par:resynchronisation}).

		Selbige Resynchronisation wurde im entwickelten
		Protokoll bzw. Authentifizierungsvorgang bis zu diesem Zeitpunkt
		nicht berücksichtigt. Nach der Identifikation dieses Defizits
		konnte bei der Implementierung nachgebessert werden.

		Alle erforderlichen Informationen waren der Standardisierung
		durch die \textit{3GPP} zu entnehmen. Mit dieser Grundlage konnten
		notwendige Parameter aus der übermittelten \textit{AUTS} extrahiert
		und weiterverarbeitet werden, sodass die \textit{SQN} erfolgreich
		resynchronisiert wird. 

		Mit der Integration dieser Funktion konntem auch erfolgreich die
		Vektoren \textit{Ck}, \textit{Ik} und \textit{Kc} generiert werden.

\clearpage