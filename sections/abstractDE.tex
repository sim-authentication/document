\subsection*{Autoren}
\begin{addmargin}[1em]{2em}
Marco Heumann und Marco Schenkel
\end{addmargin}

\subsection*{Thema der Arbeit}
\begin{addmargin}[1em]{2em}
Programmierung einer SIM-Authentifizierung
\end{addmargin}

\subsection*{Stichworte}
\begin{addmargin}[1em]{2em}
Authentifizierung, SIM-Karten, UMTS
\end{addmargin}

\subsection*{Kurzzusammenfassung}
\begin{addmargin}[1em]{2em}
In dieser Arbeit wurde die Authentifizierung einer SIM-Karte
bei einem Netzprovider nachgebaut. Die Idee dahinter ist, dass
ein Hotel seinen Gästen eine SIM-Karte gibt. Diese müssen die
Karte in ihrem Zimmer in ein Kartenlesegerät stecken und bekommen
dann Zugang zum Internet. \\
In dieser Arbeit wird der UMTS Standard verwendet. Dieser benutzt
zur Authentifizierung den Milenage Algorithmus und damit verbunden
die standardisierte Blockchiffre AES.

Umgesetzt wurde in dieser Arbeit der Authentifizierungsvorgang. Dazu
wurde eine virtuelle Umgebung aufgesetzt in der die Authentifizierungsschnittstelle
läuft. Das Programm, welches in dieser Umgebung läuft und das die Berechnungen
des Milenage und AES Algorithmus durchführt, wurde in der Sprache C
geschrieben. \\
Auf der anderen Seite wurde ein Raspberry Pi verwendet, an dem ein Kartenlesegerät zum lesen der SIM-Karte
angeschlossen ist. Die SIM-Karte wird über den Linux-Treiber
PCSClite angesprochen. Ein Pythonprogramm steuert über den Treiber die Kommunikation
mit der SIM-Karte. \\
Die Kommunikation zwischen der virtuellen Umgebung und Raspberry Pi läuft über
ein Ethernetkabel und dem PPPoE-Protokoll.

Abschließend wird sowohl eine Auswertung über den Ablauf des Projektes als auch
eine Diskussion, welche mögliche Projekterweiterungen anspricht, ausgeführt.
\end{addmargin}