%% Hier alle Abkürzungen eintragen und mit \ac{Abk.} benutzen.
%% ac{ABK} setzt die Abkürzung automatisch 
%% acl{Abk} schreibt die lange Version
%% wenn an einen /ac Befehl ein p angehängt wird /acp wird die Pluralform
% verwendet
\acro{3GPP}{3rd Generation Partnership Project}
\acro{SIM}{Subscriber Identity Module}
\acro{USIM}{Universal Subscriber Identity Module}
\acro{IMSI}{International Mobile Subscriber Identity}
\acro{GSM}{Global System for Mobile communications}
\acro{UMTS}{Universal Mobile Telecommunications System}		-
\acro{PIN}{Personal Identification Number}
\acro{EF}{Elementary File}
\acro{AuC}{Authentication Center}
\acro{UE}{User Equipment}
\acro{RES}{Response to Challenge}
\acro{AMF}{Authentication Management Field}
\acro{SQN}{Sequence Number}
\acro{AES}{Advanced Encryption Standard}
\acro{OP}{Operator Configuration Field}
\acro{OPc}{Operator Configuration Field encrypted}
\acro{K}{Subscriber Key}
\acro{RAND}{Random Challenge}
\acro{PPP}{Point To Point Protocol}
\acro{PPPoE}{Point to Point Protocol over Ethernet}
\acro{ISP}{Internet Service Provider}
\acro{LAN}{Local Area Network}
\acro{SLIP}{Serial Line Internet Protocol}
\acro{LCP}{Link Control Protocol}
\acro{NCP}{Network Control Protocol}
\acro{HDLC}{High Level Data Link Control}
\acro{IANA}{Internet Assigned Numbers Authority}
\acro{PAP}{Password Authentication Protocol}
\acro{CHAP}{Challenge Handshake Authentication Protocol}
\acro{NCP}{Network Control Protocol}
\acro{IPCP}{Internet Protocol Control Protocol}