%% Hier alle Abkürzungen eintragen und mit \ac{Abk.} benutzen.
%% ac{ABK} setzt die Abkürzung automatisch 
%% acl{Abk} schreibt die lange Version
%% wenn an einen /ac Befehl ein p angehängt wird /acp wird die Pluralform
% verwendet
\acro{3GPP}{3rd Generation Partnership Project}
\acro{AES}{Advanced Encryption Standard}
\acro{AMF}{Authentication Management Field}
\acro{AP}{Authentifizierungspunkt}
\acro{APDU}{Application Protocol Data Unit}
\acro{AuC}{Authentication Center}
\acro{AV}{Authentifizierungsvektor}
\acro{BSC}{Base Station Controller}
\acro{BSS}{Base Station System}
\acro{CHAP}{Challenge Handshake Authentication Protocol}
\acro{DES}{Data Encryption Standard}
\acro{DF}{Dedicated File}
\acro{EEPROM}{Electrically Erasable Programmable Read-Only Memory}
\acro{EF}{Elementary File}
\acro{EIR}{Equipment Identity Register}
\acro{GSM}{Global System for Mobile communications}
\acro{HDLC}{High Level Data Link Control}
\acro{HLR}{Home Location Register}
\acro{IANA}{Internet Assigned Numbers Authority}
\acro{ICC}{Integrated Chip Card}
\acro{ICCID}{Integrated Circuit Card Identifier}
\acro{IEEE}{Institute of Electrical and Electronics Engineers}
\acro{IFD}{Interface Device}
\acro{IMSI}{International Mobile Subscriber Identity}
\acro{IPCP}{Internet Protocol Control Protocol}
\acro{ISP}{Internet Service Provider}
\acro{K}{Subscriber Key}
\acro{LAI}{Local Area Identity}
\acro{LAN}{Local Area Network}
\acro{LCP}{Link Control Protocol}
\acro{MCC}{Mobile Country Code}
\acro{MF}{Master File}
\acro{MNC}{Mobile Network Code}
\acro{MS}{Mobile Station}
\acro{MSC}{Mobile Switching Center}
\acro{MSIN}{Mobile Subscriber Identification Number}
\acro{NAT}{Network Address Translation}
\acro{NCP}{Network Control Protocol}
\acro{NIST}{National Institute of Standards and Technology}
\acro{NSA}{National Security Agency}
\acro{OP}{Operator Configuration Field}
\acro{OPc}{Operator Configuration Field encrypted}
\acro{PAP}{Password Authentication Protocol}
\acro{PC/SC}{Personal Computer / Smart Card}
\acro{PIN}{Personal Identification Number}
\acro{PPP}{Point To Point Protocol}
\acro{PPPoE}{Point to Point Protocol over Ethernet}
\acro{PUK}{Personal Unlocking Key}
\acro{RNC}{Radio Network Controller}
\acro{RAND}{Random Challenge}
\acro{RES}{Response to Challenge}
\acro{SGSN}{Serving GPRS Support Node}
\acro{SIM}{Subscriber Identity Module}
\acro{SLIP}{Serial Line Internet Protocol}
\acro{SMS}{Short Messaging Service}
\acro{SN}{Serving Network}
\acro{SQN}{Sequence Number}
\acro{ssh}{Secure Shell}
\acro{TMSI}{Temporary Mobile Subscriber Identity}
\acro{tty}{tele-typewriter}
\acro{UE}{User Equipment}
\acro{UICC}{Universal Integrated Circuit Card}
\acro{UMTS}{Universal Mobile Telecommunications System}		-
\acro{USIM}{Universal Subscriber Identity Module}
\acro{VLR}{Visitor Location Register}